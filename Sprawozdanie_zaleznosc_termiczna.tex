\documentclass{article}
\usepackage[utf8]{inputenc}
\usepackage[T1]{fontenc}
\usepackage[polish]{babel}
\usepackage{csquotes}
%\usepackage[sorting=none,giveninits=true]{biblatex}
%\addbibresource{bibliography.bib}
\usepackage{float}
\usepackage{graphicx}
\graphicspath{{./}}
%\usepackage{multirow}
\usepackage{makecell}
\usepackage{xcolor}  % kolory motywu
\usepackage{tikz}
\usetikzlibrary{angles}
\usetikzlibrary{quotes}
\usetikzlibrary{decorations.pathreplacing}
\usetikzlibrary{calligraphy}
\usetikzlibrary{arrows.meta}
\usetikzlibrary{calc}
\usepackage{pgfplots}
\usepackage{pgfplotstable}
\pgfplotsset{compat=1.9}
\pgfkeys{/pgf/number format/.cd,1000 sep={\,}}
\pgfplotsset{major grid style={thick}}
\pgfplotsset{minor grid style={dashed}}
\usepackage{amsmath}  % równania
\usepackage{amssymb}
\usepackage{bbold}
\usepackage{physics2}  % pochodne, macierze itp
\usephysicsmodule{ab}
\usephysicsmodule{diagmat}
\usephysicsmodule{xmat}
\usephysicsmodule{nabla.legacy}
\usephysicsmodule{op.legacy}
\makeatletter
%\newcommand\vb[1]{\@ifstar\boldsymbol\mathbf{#1}}
\newcommand\vb[1]{\@ifstar\boldsymbol\mathbf{#1}}
\newcommand\va[1]{\@ifstar{\vec{#1}}{\vec{\mathrm{#1}}}}
\newcommand\vu[1]{%
	\@ifstar{\hat{\boldsymbol{#1}}}{\hat{\boldsymbol{#1}}}}
\makeatother
\usepackage{fixdif, derivative}  % pochodne
\usepackage[version=4]{mhchem}
\usepackage{siunitx}
\usepackage{booktabs}
\usepackage{caption}
\usepackage{subcaption}
\title{Zależność parametrów ogniwa fotowoltaicznego od temperatury}
\author{Jędrzej Górny, Jan Kurek, Rafał Staroszczyk}
\date{}

\setlength{\abovedisplayskip}{0pt}
\setlength{\belowdisplayskip}{0pt}
\setlength{\abovedisplayshortskip}{0pt}
\setlength{\belowdisplayshortskip}{0pt}

\newcommand{\inv}[1]{\frac{1}{#1}}

\begin{document}
	\maketitle
	\section{Metodologia badania}
	\section{Wyniki i wnioski}
	Podczas badania krzywej prądowo-napięciowej ogniwa otrzymano następujące wykresy dla różnych temperatur.
	\begin{figure}[H]
		\centering
		\begin{tikzpicture}
			\begin{axis}[scale only axis,
						 width=0.8\textwidth,
						 xlabel=$U\;\bab{\unit{\mV}}$,
						 ylabel=$I\;\bab{\unit{\mA}}$,
						 grid=both,
						 xmin=0,
						 xmax=2200,
						 ymin=0,
						 ymax=130,
						 xtick distance=500,
						 ytick distance=50,
						 minor tick num=4,
						 legend pos=south west]
				\addplot [color=red  , smooth, mark=*, mark size=1.5pt] table [x=UmV, y=ImA] {temp_25_UI_sorted.txt};
				\addplot [color=green, smooth, mark=*, mark size=1.5pt] table [x=UmV, y=ImA] {temp_40_UI_sorted.txt};
				\addplot [color=blue , smooth, mark=*, mark size=1.5pt] table [x=UmV, y=ImA] {temp_50_UI_sorted.txt};
				\addplot [color=black, smooth, mark=*, mark size=1.5pt] table [x=UmV, y=ImA] {temp_60_UI_sorted.txt};
				\legend{$T=\qty{25}{\degreeCelsius}$,
						$T=\qty{40}{\degreeCelsius}$,
						$T=\qty{50}{\degreeCelsius}$,
						$T=\qty{60}{\degreeCelsius}$}
			\end{axis}
		\end{tikzpicture}
		\caption{Zależność prądowo-napięciowa od temperatury}
		\label{fig:i(u)}
	\end{figure}
	Z otrzymanych danych można obliczyć zależność mocy od napięcia według wzoru $P=UI$.
	\begin{figure}[H]
		\centering
		\begin{tikzpicture}
			\begin{axis}[scale only axis,
				width=0.8\textwidth,
				xlabel=$U\;\bab{\unit{\mV}}$,
				ylabel=$P\;\bab{\unit{\mW}}$,
				grid=both,
				xmin=0,
				xmax=2200,
				ymin=0,
				ymax=180,
				xtick distance=500,
				ytick distance=50,
				minor tick num=4,
				legend pos=north west]
				\addplot [color=red  , smooth, mark=*, mark size=1.5pt] table [x=UmV, y=PmW] {temp_25_PU_sorted.txt};
				\addplot [color=green, smooth, mark=*, mark size=1.5pt] table [x=UmV, y=PmW] {temp_40_PU_sorted.txt};
				\addplot [color=blue , smooth, mark=*, mark size=1.5pt] table [x=UmV, y=PmW] {temp_50_PU_sorted.txt};
				\addplot [color=black, smooth, mark=*, mark size=1.5pt] table [x=UmV, y=PmW] {temp_60_PU_sorted.txt};
				\legend{$T=\qty{25}{\degreeCelsius}$,
					$T=\qty{40}{\degreeCelsius}$,
					$T=\qty{50}{\degreeCelsius}$,
					$T=\qty{60}{\degreeCelsius}$}
			\end{axis}
		\end{tikzpicture}
		\caption{Zależność krzywej mocy od temperatury}
		\label{fig:p(u)}
	\end{figure}
	Z wykresu \ref{fig:p(u)} można odczytać parametry ogniwa w punkcie maksymalnej mocy:
	\begin{table}[h]
		\centering
		\begin{tabular}{
			S[table-column-width=1.5cm]
			S[table-column-width=1.5cm]
			S[table-column-width=1.5cm]
			S[table-column-width=1.5cm]
			S[table-column-width=1.5cm]
			S[table-column-width=1.5cm]
			S[table-column-width=1.5cm]}
			\toprule
			{\makecell{Temperatura \\ $\bab{\unit{\degreeCelsius}}$}} & 
			{\makecell{$U_{OC}$ \\ $\bab{\unit{\mV}}$}} & 
			{\makecell{$I_{SC}$ \\ $\bab{\unit{\mA}}$}} & 
			{\makecell{$U_{MPP}$ \\ $\bab{\unit{\mV}}$}} & 
			{\makecell{$I_{MPP}$ \\ $\bab{\unit{\mA}}$}} & 
			{\makecell{FF \\ $\bab{\unit{\percent}}$}} & 
			{\makecell{$P_{MPP}$ \\ $\bab{\unit{\mW}}$}} \\
			\midrule
			%T   Uoc      Isc     Umpp     Impp    FF     Pmpp
			25 & 2029.2 & 112.5 & 1581.6 & 104.9 & 73 & 165.9 \\
			40 & 1898.3 & 116.9 & 1495.9 & 104.9 & 71 & 156.9 \\
			50 & 1859.5 & 118.0 & 1390.7 & 108.6 & 69 & 151.0 \\
			60 & 1770.3 & 121.6 & 1317.2 & 110.5 & 68 & 145.6 \\
			\bottomrule
		\end{tabular}
		\caption{Parametry ogniwa w PMM dla różnych temperatur}
	\end{table}
	
	\pagebreak
	Wraz ze wzrostem temperatury występuje spadek napięcia układu otwartego i punktu maksymalnej mocy, współczynnika wypełnienia oraz mocy maksymalnej. Zwiększa się jednak prąd obwodu zamkniętego i punktu maksymalnej mocy. Wszystkie zależności są w przybliżeniu liniowe w badanym zakresie. 
	\begin{figure}[H]
		\centering
		\begin{subfigure}{0.45\textwidth}
			\centering
			\begin{tikzpicture}
				\begin{axis}[scale only axis,
					width=0.7\textwidth,
					xlabel=$T\;\bab{\unit{\degreeCelsius}}$,
					ylabel=$U_{OC}\;\bab{\unit{\mV}}$,
					grid=both,
					xmin=15,
					xmax=70,
					ymin=1200,
					ymax=2200,
					xtick distance=20,
					ytick distance=400,
					minor tick num=3]
					\addplot coordinates {
					(25, 2029.2)
					(40, 1898.3)
					(50, 1859.5)
					(60, 1770.3)};
				\end{axis}
			\end{tikzpicture}
			\caption{Napięcie obwodu otwartego}
		\end{subfigure}
		\begin{subfigure}{0.45\textwidth}
			\centering
			\begin{tikzpicture}
				\begin{axis}[scale only axis,
					width=0.7\textwidth,
					xlabel=$T\;\bab{\unit{\degreeCelsius}}$,
					ylabel=$U_{MPP}\;\bab{\unit{\mV}}$,
					grid=both,
					xmin=15,
					xmax=70,
					ymin=1200,
					ymax=2200,
					xtick distance=20,
					ytick distance=400,
					minor tick num=3]
					\addplot coordinates {
					(25, 1581.6)
					(40, 1495.9)
					(50, 1390.7)
					(60, 1317.2)};
				\end{axis}
			\end{tikzpicture}
			\caption{Napięcie MPP}
		\end{subfigure}
		\begin{subfigure}{0.45\textwidth}
			\centering
			\begin{tikzpicture}
				\begin{axis}[scale only axis,
					width=0.7\textwidth,
					xlabel=$T\;\bab{\unit{\degreeCelsius}}$,
					ylabel=$I_{SC}\;\bab{\unit{\mA}}$,
					grid=both,
					xmin=15,
					xmax=70,
					ymin=100,
					ymax=130,
					xtick distance=20,
					ytick distance=10,
					minor tick num=3]
					\addplot coordinates {
						(25, 112.5)
						(40, 116.9)
						(50, 118.0)
						(60, 121.6)};
				\end{axis}
			\end{tikzpicture}
			\caption{Prąd obwodu zamkniętego}
		\end{subfigure}
		\begin{subfigure}{0.45\textwidth}
			\centering
			\begin{tikzpicture}
				\begin{axis}[scale only axis,
					width=0.7\textwidth,
					xlabel=$T\;\bab{\unit{\degreeCelsius}}$,
					ylabel=$I_{MPP}\;\bab{\unit{\mA}}$,
					grid=both,
					xmin=15,
					xmax=70,
					ymin=100,
					ymax=130,
					xtick distance=20,
					ytick distance=10,
					minor tick num=3]
					\addplot coordinates {
					(25, 104.9)
					(40, 104.9)
					(50, 108.6)
					(60, 110.5)};
				\end{axis}
			\end{tikzpicture}
			\caption{Prąd MPP}
		\end{subfigure}
		\begin{subfigure}{0.45\textwidth}
			\centering
			\begin{tikzpicture}
				\begin{axis}[scale only axis,
					width=0.7\textwidth,
					xlabel=$T\;\bab{\unit{\degreeCelsius}}$,
					ylabel=$FF\;\bab{\unit{\percent}}$,
					grid=both,
					xmin=15,
					xmax=70,
					ymin=60,
					ymax=80,
					xtick distance=20,
					ytick distance=10,
					minor tick num=4]
					\addplot coordinates {
					(25, 73)
					(40, 71)
					(50, 69)
					(60, 68)};
				\end{axis}
			\end{tikzpicture}
			\caption{Współczynnik wypełnienia}
		\end{subfigure}
		\begin{subfigure}{0.45\textwidth}
			\centering
			\begin{tikzpicture}
				\begin{axis}[scale only axis,
					width=0.7\textwidth,
					xlabel=$T\;\bab{\unit{\degreeCelsius}}$,
					ylabel=$P_{MPP}\;\bab{\unit{\mW}}$,
					grid=both,
					xmin=15,
					xmax=70,
					ymin=140,
					ymax=180,
					xtick distance=20,
					ytick distance=10,
					minor tick num=4]
					\addplot coordinates {
					(25, 165.9)
					(40, 156.9)
					(50, 151.0)
					(60, 145.6)};
				\end{axis}
			\end{tikzpicture}
			\caption{Maksymalna moc}
		\end{subfigure}
		\caption{Zmiany parametrów z temperaturą}
	\end{figure}
	
	\pagebreak
	\section{Podsumowanie}
\end{document}