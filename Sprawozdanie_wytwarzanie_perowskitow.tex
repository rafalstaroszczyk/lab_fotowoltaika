\documentclass{article}
\usepackage[utf8]{inputenc}
\usepackage[T1]{fontenc}
\usepackage[polish]{babel}
\usepackage{csquotes}
%\usepackage[sorting=none,giveninits=true]{biblatex}
%\addbibresource{bibliography.bib}
\usepackage{float}
\usepackage{graphicx}
\graphicspath{{./}}
%\usepackage{multirow}
\usepackage{makecell}
\usepackage{xcolor}  % kolory motywu
\usepackage{tikz}
\usetikzlibrary{angles}
\usetikzlibrary{quotes}
\usetikzlibrary{decorations.pathreplacing}
\usetikzlibrary{calligraphy}
\usetikzlibrary{arrows.meta}
\usetikzlibrary{calc}
\usepackage{pgfplots}
\usepackage{pgfplotstable}
\pgfplotsset{compat=1.9}
\pgfkeys{/pgf/number format/.cd,1000 sep={\,}}
\pgfplotsset{major grid style={thick}}
\pgfplotsset{minor grid style={dashed}}
\usepackage{amsmath}  % równania
\usepackage{amssymb}
\usepackage{bbold}
\usepackage{physics2}  % pochodne, macierze itp
\usephysicsmodule{ab}
\usephysicsmodule{diagmat}
\usephysicsmodule{xmat}
\usephysicsmodule{nabla.legacy}
\usephysicsmodule{op.legacy}
\makeatletter
%\newcommand\vb[1]{\@ifstar\boldsymbol\mathbf{#1}}
\newcommand\vb[1]{\@ifstar\boldsymbol\mathbf{#1}}
\newcommand\va[1]{\@ifstar{\vec{#1}}{\vec{\mathrm{#1}}}}
\newcommand\vu[1]{%
	\@ifstar{\hat{\boldsymbol{#1}}}{\hat{\boldsymbol{#1}}}}
\makeatother
\usepackage{fixdif, derivative}  % pochodne
\usepackage[version=4]{mhchem}
\usepackage{siunitx}
\usepackage{booktabs}
\usepackage{caption}
\usepackage{subcaption}
\title{Wytwarzanie perowskitów}
\author{Jędrzej Górny, Jan Kurek, Rafał Staroszczyk}
\date{}

\setlength{\abovedisplayskip}{0pt}
\setlength{\belowdisplayskip}{0pt}
\setlength{\abovedisplayshortskip}{0pt}
\setlength{\belowdisplayshortskip}{0pt}

\newcommand{\inv}[1]{\frac{1}{#1}}

\begin{document}
	\maketitle
	\section{Materiały perowskitowe}
	Oryginalnym perowskitem był tytanian wapnia \ce{CaTiO3}. Nazwę tą rozwinięto do szerszej grupy związków o wzorze ogólnym \ce{ABX3}. Możliwych jest kilka konfiguracji ze względu na wartościowości pierwiastków, ale w fotowoltaice najważniejsze jest podgrupa o wzorze ogólnym \ce{A^{+}B^{2+}X^{-}3}. Typowymi jonami są dla \ce{A^{+}}: \ce{MA^{+}}, \ce{FA^{+}}, \ce{EA^{+}}, \ce{Cs^{+}} i \ce{Rb^{+}}; dla \ce{B^{2+}} dominującym jest \ce{Pb^{2+}}; dla {\ce{X^{-}}: \ce{I^{-}}, \ce{Br^{-}} i \ce{Cl^{-}}. Podczas zajęć wykonano ogniwa na bazie perowskitu \ce{Cs_{x}FA_{1-(x+y)}MA_{y}PbBr_{x}I_{3-x}}}.
	\section{Budowa ogniwa perowskitowego}
	\section{Wytwarzanie ogniwa perowskitowego}
	\section{Charakterystyka perowskitu}
\end{document}